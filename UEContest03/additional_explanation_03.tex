\documentclass[12pt]{jarticle}

\usepackage{cases}
\usepackage{url}
\usepackage[left=2cm, right=2cm, top=3cm, bottom=3cm]{geometry}
%\usepackage{amsmath}
%\usepackage{amssymb}
\usepackage{amsfonts}
\usepackage{here}

\title{UEContest 003 補足説明}
\author{grobner, maccha, tonphy}
\date{2020年5月6日}

\begin{document}

\maketitle

公式の解説に不足している点を補足説明します。


\section{ABC108B - Ruined Square}

\subsection{問題文(要約)}
xy平面上に正方形があり、各頂点の座標を反時計回りに順に、$(x_1, y_1), (x_2, y_2), (x_3, y_3), (x_4, y_4)$とする。$x_1, y_1, x_2, y_2$が与えられるので、$x_3, y_3, x_4, y_4$を求めて、出力せよ。

なお、$x_3, y_3, x_4, y_4$は一意に定まる整数である。

\subsection{制約}
\begin{itemize}
	\item $|x_1|, |y_1|, |x_2|, |y_2| \leq 100$
	\item $(x_1, y_1) \neq (x_2, y_2)$
	\item 入力はすべて整数
\end{itemize}

\subsection{補足説明}
解説によれば、$x_3, y_3, x_4, y_4$は、それぞれ
\begin{numcases}
	{}
	x_3 = x_2 + y_1 - y_2 \label{C_x}& \\
	y_3 = x_2 - x_1 + y_2 \label{C_y}&
\end{numcases}
\begin{numcases}
	{}
	x_4 = x_1 + y_1 - y_2 \label{D_x}& \\
	y_4 = x_2 - x_1 + y_1 \label{D_y}&
\end{numcases}
である。このことを示したい。ここでは、高校数学で習う「複素数」を用いる。

一般に、複素数平面において、点B$(b)$を点A$(a)$を中心に$\theta$だけ回転させてできた点C$(c)$は、次式で表される。
\begin{eqnarray*}
	c = (b - a)(\cos\theta + i\sin\theta) + a
\end{eqnarray*}
なお$\overrightarrow{\mathrm{AC}}$は、$\overrightarrow{\mathrm{AB}}$を点Aを中心に$\theta$だけ回転させてできるベクトルである。
\newline

正方形の各頂点を、複素数平面上で反時計回りにA$(x_1, y_1)$, B$(x_2, y_2)$, C$(x_3, y_3)$, D$(x_4, y_4)$とする。このとき、次のことが常に成り立つ。
\begin{itemize}
	\item $\overrightarrow{\mathrm{AB}}$を、点$\mathrm{A}$を中心に$\frac{\pi}{2}$だけ回転させてできるベクトルは、$\overrightarrow{\mathrm{AD}}$である。
	\item $\overrightarrow{\mathrm{DC}} = \overrightarrow{\mathrm{AB}}$ゆえ、$\overrightarrow{\mathrm{OC}} = \overrightarrow{\mathrm{OD}} + \overrightarrow{\mathrm{AB}}$
\end{itemize}

まず、点Dを示す複素数$(x_4 + y_4i)$を求める。$\overrightarrow{\mathrm{AD}}$は$\overrightarrow{\mathrm{AB}}$を、点$\mathrm{A}$を中心に$\frac{\pi}{2}$だけ回転させてできるベクトルである。
\begin{eqnarray*}
	x_4 + y_4i & = & \{(x_2 - x_1) + (y_2 - y_1)i\}\left(\cos\frac{\pi}{2} + i\sin\frac{\pi}{2}\right) + (x_1 + y_1i) \\
	& = & (x_1 + y_1 - y_2) + (x_2 - x_1 + y_1)i
\end{eqnarray*}
したがって、式(\ref{D_x}), (\ref{D_y})と一致する。

次に、点Cを示す複素数$(x_3 + y_3i)$を求める。$\overrightarrow{\mathrm{DC}} = \overrightarrow{\mathrm{AB}}$であるから、
\begin{eqnarray*}
	\overrightarrow{\mathrm{OC}} & = & \overrightarrow{\mathrm{OD}} + \overrightarrow{\mathrm{DC}} \\
	& = & \overrightarrow{\mathrm{OD}} + \overrightarrow{\mathrm{AB}}
\end{eqnarray*}
となる。よって、
\begin{eqnarray*}
	x_3 + y_3i & = & \{(x_2 - x_1) + (y_2 - y_1)i\} + (x_4 + y_4i) \\
	& = & \{(x_2 - x_1) + (y_2 - y_1)i\} + (x_1 + y_1 - y_2) + (x_2 - x_1 + y_1)i \\
	& = & (x_2 + y_1 - y_2) + (x_2 - x_1 + y_2)i
\end{eqnarray*}
これは、式(\ref{C_x}), (\ref{C_y})と一致する。

まとめると、
\begin{numcases}
{}
x_3 = x_2 + y_1 - y_2 \nonumber& \\
y_3 = x_2 - x_1 + y_2 \nonumber&
\end{numcases}
\begin{numcases}
{}
x_4 = x_1 + y_1 - y_2 \nonumber& \\
y_4 = x_2 - x_1 + y_1 \nonumber&
\end{numcases}

\newpage

\section{ABC159C - Maximum Volume}

\subsection{問題文(要約)}
縦、横、高さの長さの合計が$L (>0)$の直方体としてありうる体積の最大値を求めよ。

\subsection{制約}
\begin{itemize}
	\item $1 \leq L \leq 1000$
	\item $L \in \mathbb{Z}$
\end{itemize}

\subsection{補足説明}
公式の解説によれば、相加平均と相乗平均の関係により、直方体の体積の最大値は$\frac{L^3}{27}$であることが分かった。ここでは、高校数学の「微分」を用いて直方体の最大値を求める方法を紹介したい。
\newline

直方体の縦、横、高さの長さを$a, b, c$とすれば
\begin{numcases}
	{}
	0 < a, b, c < L \nonumber& \\
	a + b + c = L \label{abcL}&
\end{numcases}
の制約に縛られる。直方体の体積$V$は、
\begin{eqnarray*}
	V = abc
\end{eqnarray*}
と表される。式(\ref{abcL})より、$a = L - b - c$であるから
\begin{eqnarray*}
	V & = & (L - b - c) \times bc \\
	& = & -cb^2 + (L - c)cb \\
	& = & -c\left(b - \frac{L - c}{2}\right)^2 + \frac{(L - c)^2c}{4}
\end{eqnarray*}
$b = \frac{L - c}{2}$のとき、最大値$V = \frac{(L - c)^2c}{4}$をとる。

ここで、$W = \frac{(L - c)^2c}{4}$とおくと、
\begin{eqnarray*}
	W & = & \frac{(L - c)^2c}{4} \\
	& = & \frac{c^3 - 2Lc^2 + L^2c}{4}
\end{eqnarray*}
と変形できる。
\begin{eqnarray*}
	\frac{\mathrm{d}W}{\mathrm{d}c} & = & \frac{3c^2 - 4Lc + L^2}{4} \\
	& = & \frac{(c - L)(3c - L)}{4}
\end{eqnarray*}

$0 < c < L$において、増減表は次のようになる。
\begin{table}[H]
	\begin{center}
		\begin{tabular}{|c||c|c|c|c|c|} \hline
			$c$ & $(0)$ & $\cdots$ & $L / 3$ & $\cdots$ & $(L)$ \\ \hline \hline
			$\mathrm{d}W / \mathrm{d}c$ &  & + & 0 & - & \\ \hline
			$W$ & (0) & $\nearrow$ & Max & $\searrow$ & (0) \\ \hline
		\end{tabular}
	\end{center}
\end{table}

よって、$c = \frac{L}{3}$で$W$は最大となる。つまり、$b = \frac{L}{3}, c = \frac{L}{3}$で直方体の体積$V$は最大となる。このとき、$a = \frac{L}{3}$である。

したがって、$a = b = c = \frac{L}{3}$のとき、直方体の体積$V$は
\begin{eqnarray*}
	V = \frac{L^3}{27}
\end{eqnarray*}
で最大となる。


\begin{thebibliography}{5}
	\bibitem{ABC108B_exp} AtCoder, "ARC 102 解説", \url{https://img.atcoder.jp/arc102/editorial.pdf}, latest accessed on 2020/5/2
	\bibitem{ABC159C_exp} AtCoder, "ABC 159 解説", \url{https://img.atcoder.jp/abc159/editorial.pdf}, latest accessed on 2020/5/2
\end{thebibliography}
\end{document}